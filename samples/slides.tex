\documentclass{beamer}

\usepackage{graphicx}
    \graphicspath{{./imgs}}
\usepackage{eso-pic}


\title{There Is No Largest Prime Number}
\author[Euclid]{Euclid of Alexandria}
\date[ISPN ’80]{27th International Symposium of Prime Numbers}
\institute[Binghamton University]{State University of New York at Binghamton}
\titlegraphic{\includegraphics[scale=0.15]{BU-LockupH-WatsonCollege-342.png}}

\usetheme{Binghamton}
\usecolortheme{bearcat}

\begin{document}

\begin{frame}[plain]
    \titlepage
\end{frame}

% This snippet adds the "B" logomark to the title bar on subsequent slides. 
% Please add the following required packages to your document preamble:
% \usepackage{graphicx}
% \usepackage{eso-pic}
\newcommand\AtPagemyUpperRight[1]{\AtPageLowerLeft{%
\put(\LenToUnit{0.91\paperwidth},\LenToUnit{0.88\paperheight}){#1}}}
\AddToShipoutPictureFG{
\AtPagemyUpperRight{{\includegraphics[scale=0.1]{Binghamton-Logo-Icon.png}}}
}%

\begin{frame}
    \frametitle{Outline}
    \tableofcontents
\end{frame}

\section{Introduction}

\begin{frame}
\frametitle{What Are Prime Numbers?}
\begin{block}{Prime Number}
A number that has exactly two divisors.
\end{block}
\begin{exampleblock}{Examples}
\begin{itemize}
\item 2 is prime (two divisors: 1 and 2).
\item 3 is prime (two divisors: 1 and 3).
\item 4 is not prime (\alert{three} divisors: 1, 2, and 4).
\end{itemize}
\end{exampleblock}
\begin{alertblock}{Danger!}
Some people say 0 and 1 are prime. 
\begin{itemize}
    \item[\textrightarrow] They are not, because they have only zero divisors and one divisor, respectively.
\end{itemize}
\end{alertblock}
\end{frame}

\section{Proof}

\begin{frame}
\frametitle{There Is No Largest Prime Number}
\framesubtitle{The proof uses \textit{reductio ad absurdum}.}
\begin{theorem}
There is no largest prime number.
\end{theorem}
\begin{proof}
\begin{enumerate}
\item<1-| alert@1> Suppose $p$ were the largest prime number.
\item<2-> Let $q$ be the product of the first $p$ numbers.
\item<3-> Then $q+1$ is not divisible by any of them.
\item<4-> Thus $q+1$ is also prime and greater than $p$.\qedhere
\end{enumerate}
\end{proof}
\end{frame}

\section{Future Work}

\begin{frame}
\frametitle{What’s Still To Do?}
\begin{itemize}
\item Answered Questions
\begin{itemize}
\item How many primes are there?
\end{itemize}
\item Open Questions
\begin{itemize}
\item Is every even number the sum of two primes?
\end{itemize}
\end{itemize}
\end{frame}

\end{document}
